\section{Axioma de elección y lema de Zorn}
En la siguiente sección sólo se presentaran dos enunciados sumamente importantes en la teoría de conjuntos. Esto con el objetivo de no dejar por fuera dos enunciados que resultan cruciales en la matemática y su relación que ha dado tanto de qué hablar en la historia de la matemática.
\begin{defn}
    En el sistema axiomatico Zermelo-Fränkel (ZF) podemos aceptar un axioma más: el axioma de elección, el cuál consiste de lo siguiente:
    \[(\forall x)(((\forall y\in x)(y\neq\nt)\land(\forall y,z\in x)(y\cap z=\nt))\Longrightarrow(\exists z)(\forall y\in x)(\exists!w)(w\in z\land w\in y))\]
    Intuitivamente, si $x$ es una familia de conjuntos no vacíos disjuntos dos a dos, entonces se puede escoger un elemento de cada miembro de $x$.
\end{defn}
\begin{defn}
    El siguiente enunciado es conocido como \textit{el lema de Zorn}:\\
    Todo conjunto parcialmente ordenado, no vacío, en el que toda cadena tiene una cota superior, contiene un elemento maximal.
\end{defn}
\begin{thm} (ZF)\\
    El axioma de elección es equivalente al lema de Zorn.
\end{thm}