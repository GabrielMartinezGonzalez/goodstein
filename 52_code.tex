\section{Código que cálcula sucesiones de Goodstein}
A continuación presentaremos un código en \textit{Python} que se desarrollo como primer acercamiento a las sucesiones de Goodstein, de forma en que se pueda visualizar la velocidad con la que crecen estas mismas:
\clearpage
\begin{verbatim}
import math as mt

def base_n(n,p):
    p1=p
    p_base=[]
    if p%n!=0:
        p_base.append(p%n)
        p1=p1-p%n
        if p1!=0:
            p_base.append("+")
        else:
            return p_base
    i=1
    while i!=0:
        con=0
        while mt.pow(n,con+1)<=p1:
            con=con+1
        kon=1
        while (kon+1)*mt.pow(n,con)<=p1:
            kon=kon+1
        if kon!=1:
            p_base=p_base+[kon,"*",n,"^",con]
        else:
            p_base=p_base+[n,"^",con]
        p1=p1-kon*mt.pow(n,con)
        if p1!=0:
            p_base.append("+")
        else:
            return p_base

def salto(p_base,n,p):
    p_aux=p_base
    if p_base==[]:
        return 0
    if len(p_base)==1:
        return p_base[0]
    if p_base[1]=="+":
        p_aux=p_aux[2:]
        return int(p_base[0]+salto(p_aux,n,p))
    if p_base[1]=="^":
        p_aux=p_aux[4:]
        return int(mt.pow((p_base[0]+1),
                salto(base_n(n,p_base[2]),n,p))+salto(p_aux,n,p))
    if p_base[1]=="*":
        p_aux=p_aux[6:]
        return int(p_base[0]*mt.pow((p_base[2]+1),
                salto(base_n(n,p_base[4]),n,p))+salto(p_aux,n,p))

def goodstein(p,i):
    if i==1:
        return p
    a=goodstein(p,i-1)
    if a>0:
        return salto(base_n(i,a),i,p)-1
    return 0

def sucesion_g(p,k):
    a=[]
    for i in range(k):
        a.append(goodstein(p,i+1))
    return a
\end{verbatim}